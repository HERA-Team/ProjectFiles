\documentclass{article}
\usepackage{float,amsmath}
\usepackage{graphicx}
\usepackage{color}
\usepackage[letterpaper,margin=1in]{geometry}
\usepackage{hyperref}

%\setlength{\textwidth}{6.5in}

\begin{document}

\author{HERA}
\title{Roadmap for HERA Network Configuration}
\maketitle

\section{Introduction}
HERA is an international experiment to detect and characterize the Epoch of Reionization (EOR).  The telescope is located at the South African SKA site in the Karoo
Astronomy Reserve.  This brief summarizes the overall network configuration for HERA.

1 - bandwidth in initial stage

2 - network reconfigure (prior to CMC move)
    - split network container/KAPB?
    - how many public-facing ports in each (order dozen)
    - can we handle self-hosting, what entailed

3 - outbound traffic
    - Need requirement document to send to Jeremy Main, Peter McF
        - how much BW and when
        - preliminary Penn end-point
        - final NRAO end-point:  get info

%\begin{figure}[H]
%\includegraphics[width=0.8\textwidth]{FILE.pdf}
%\centering
%\caption{CAPTION}
%\label{fig:FIGURE}
%\end{figure}

\end{document}